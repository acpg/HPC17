\documentclass[12pt,letterpaper]{article}
\usepackage[utf8]{inputenc}
\usepackage[english]{babel}
\usepackage[left=1in,right=1in,top=1in,bottom=1in]{geometry}
\usepackage{amsmath}
\usepackage{amsfonts}
\usepackage{amssymb}
\usepackage{graphicx}

\usepackage{pgf,tikz}
\usepackage{mathrsfs}
\usetikzlibrary{arrows}

\usepackage{listings}

\definecolor{darkgreen}{rgb}{0,0.5,0}
\definecolor{orange}{rgb}{1,0.5,0}
\lstset{language=C,
basicstyle=\ttfamily\scriptsize,
directivestyle=\color{red},
keywordstyle=\color{blue},
commentstyle=\color{darkgreen},
stringstyle=\color{orange},
breaklines=true,
mathescape}

\title{Quadtree/Octree Distribution of Data using p4est}
\author{Ana C. Perez-Gea}
\begin{document}
\maketitle

In this project we see the advantages of using mesh refinement to process massive amounts of data. When there are a large number of data points that a computer needs to process, it can be problematic when the computer does not have enough memory to load and process all data points at the same time. In this project I simulate a large number of data points and use quadtrees or octrees to partition the data and have different processors working with differents parts of the data.  In particular, I use the package p4est in the C programming language to do the mesh refinement and load data into different processors.

\section{C Code}

\lstinputlisting{p4est_data.c}

\end{document}